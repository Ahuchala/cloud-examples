\documentclass{article}

\usepackage[utf8]{inputenc}
\usepackage{sagetex}

\begin{document}

\section{This is a test}

Testing $\frac{1}{178} = \sage{n(1/178)}$.

\section{plotting}

is always fun ...

\sageplot[width=.8\textwidth]{plot(sin(x), (x,-10,10))}

\section{This is a test}

Testing $(1-x-x^2)^3 = \sage{((1-x-x^2)^2).expand()}$.

Using Sage\TeX, one can use Sage to compute things and put them into
your \LaTeX{} document. For example, there are
$\sage{number_of_partitions(1269)}$ integer partitions of $1269$.
You don't need to compute the number yourself, or even cut and paste
it from somewhere.

Here's some Sage code:

\begin{sageblock}
f(x) = exp(x) * sin(2*x)
\end{sageblock}

The second derivative of $f$ is

\[
  \frac{\mathrm{d}^{2}}{\mathrm{d}x^{2}} \sage{f(x)} =
  \sage{diff(f, x, 2)(x)}.
\]

Here's a plot of $f$ from $-1$ to $1$:

\sageplot[width=.8\textwidth]{plot(f, -1, 1)}

\end{document}
